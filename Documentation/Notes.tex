\documentclass{article}

\usepackage{mathtools}

\newcommand{\xv}{x_\text{v}}
\newcommand{\xvdot}{\dot{x}_\text{v}}
\newcommand{\uv}{u_\text{v}}
\newcommand{\ka}{k_\text{a}}
\newcommand{\Kx}{K_x}
\newcommand{\KF}{K_F}

\title{Exploring implications of passivity of the velocity-force response of the actuator}
\author{M. V. Sivaselvan}
\date{February 05, 2024}

\begin{document}

\maketitle

The actuator is represented by the model
\begin{equation}
	\begin{aligned}
		\dot{F} &= -\beta F + d\xv - \ka v \\
		\xvdot  &= -\alpha \xv + \alpha \uv
	\end{aligned}
	\label{eq:ActuatorModel}
\end{equation}
with the feedback law
\begin{equation}
	\uv = u - \Kx x - \KF F
	\label{eq:ActuatorFeedback}
\end{equation}

Questions:
\begin{enumerate}
	\item Under what conditions is the $v$--$F$ response (with $u = 0$)of the actuator passive?
	\item Suppose the actuator $v$--$F$ response is put in feedback with a mass (as shown in Figure XXX).
		If the actuator is passive, this feedback loop is stable.
		Stability of this loop can also be analyzed using root-locus plots (varying $\Kx$ and $\KF$).
		How does stability established in this way compare with stability implied by actuator passivity?
	\item How can model variability, for example in valve dynamics, be taken into accout in this analysis?
\end{enumerate}

\end{document}